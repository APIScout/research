To evaluate the performance of our information retrieval system, we used a ground truth.
Our ground truth consists in a textual file in which we have a query and the expected result.
Some examples are depicted in Figure~\ref{fig:ground-truth}.
\begin{figure}
    \centering
    \begin{verbatim}
         ...
         aws iot service on secure tunneling
         AWS IoT Secure Tunneling

         driving license registration service
         Transport Department

         sentiment analysis service api
         Text Analytics & Sentiment Analysis API | api.text2data.com

         email mailbox checker
         MailboxValidator Free Email Checker
         ...
    \end{verbatim}
    \caption{Ground Truth}
    \label{fig:ground-truth}
\end{figure}
The file is divided into blocks of two lines.
The first line represents the query that will be vectorized and passed to the $K$-NN query.
On the other hand, the second line represents the title of the document that needs to be found in the top 5 results of the query.
The title of the document on the second line of the block has been chosen manually, this means that we read some of the API specifications and wrote a query that can be connected to that specific document. \\ \\
However, in some cases, there can be multiple documents that satisfy one query; for example, in the case of the query \("\)driving license registration service\("\).
In the database, there are several different APIs related to the query mentioned: \("\)Transport Department, Tamil Nadu\("\) and \("\)Transport Department, Haryana\("\).
In such cases, we take the first occurrence that starts with the words \("\)Transport Department\("\).
